\usepackage[utf8]{inputenc}
\usepackage[T1]{fontenc}
\usepackage{amsmath}
\usepackage{amssymb}
\usepackage{makeidx}
\usepackage{graphicx}
\usepackage[export]{adjustbox}		% TO allow better image alignment control
%\usepackage{fancyvrb}

\usepackage{pdfpages} 				% To be able to insert PDFs

%\usepackage{showframe}				%to show the margins

\usepackage{calc}					% To allow arithmetic in latex script

%\usepackage[portuguese]{babel}		% Set the language
\usepackage[english]{babel}		% Set the language
%\usepackage{indentfirst}			% Enables the identation on the first pharagraph

\usepackage{array}
\usepackage{lipsum}					% To allow one to generate random text to test
\usepackage{blindtext}				% Random Document Generator for testing
\usepackage{xcolor}					% Allows to change text color
\usepackage{setspace}				% Allows to change spacing in the document
%\usepackage[acronym]{glossaries}	% Package for glossaries related entres
%\usepackage[numbers]{natbib}		% Bibliography in IEEE Style [Numbers] 
\usepackage{natbib}
\bibliographystyle{apalike}			% Standard Bibliography style
\usepackage{tocbibind}				% Adds the Index to the Table of Contents
\usepackage{enumitem}				% Customize lists



%Path to imge load location
\graphicspath{
              {Images/}             
             }
         
% Page Formats
\usepackage[left=3.00cm, right=2.50cm, top=3.50cm, bottom=3.00cm]{geometry}

%Set Font for main documet
% Main Text & Math: Roman Type as Times New Roman
%\renewcommand{\rmdefault}{ptm}
\usepackage{mathptmx}

% Titles, Front Cover, etc. Sans Type target: Lucinda Sans Unico
% https://en.wikipedia.org/wiki/Lucida_Sans_Unicode
% As Lucida Sans Unico is not free to used in general, the default alternative is set
%\usepackage{sectsty}
\usepackage{titlesec}
%\titleformat*{\chapter}{\huge\bfseries\sffamily}


\titleformat{\chapter}[hang]
{\fontfamily{phv}\bfseries\fontsize{12}{14.4pt}}
%{\chaptertitlename \thechapter}
{\thechapter.}
{0pt}
{\fontfamily{phv}\fontsize{12}{14.4pt}\MakeUppercase}							% Chapters Allways in Uppercase
\titlespacing*{\chapter}{0pt}{8pt}{10pt}



\titleformat*{\section}{\fontsize{10}{12pt}\bfseries\sffamily\MakeUppercase}
\titleformat*{\subsection}{\fontsize{10}{12pt}\sffamily\MakeUppercase}
\titleformat*{\subsubsection}{\fontsize{10}{12pt}\bfseries\sffamily}
\titleformat*{\paragraph}{\fontsize{10}{12pt}\itshape\sffamily}
\makeatletter
\renewcommand\paragraph{\@startsection{paragraph}{4}{\z@}%
	{-2.5ex\@plus -1ex \@minus -.25ex}%
	{1.25ex \@plus .25ex}%
	{\normalfont\normalsize\bfseries}}
\makeatother


% ############################### TOC Definitions #######################
% Make the Chapters in Table of Contents all UPPERCASE



\usepackage{tocloft}
\setlength{\cftbeforechapskip}{0.1cm}

\usepackage{textcase}
\usepackage{etoolbox}
\makeatletter
\patchcmd{\l@chapter}{#1}{\MakeTextUppercase{#1}}{}{}
\patchcmd{\l@section}{#1}{\MakeTextUppercase{#1}}{}{}
\patchcmd{\l@subsection}{#1}{\MakeTextUppercase{#1}}{}{}
\makeatother

\renewcommand{\cfttoctitlefont}{\fontfamily{phv}\bfseries\fontsize{12}{14.4pt}}
\renewcommand{\cftchapleader}{\cftdotfill{\cftdotsep}}
\renewcommand\cftchapfont{\mdseries\fontsize{9}{10.8pt}}
\renewcommand\cftsecfont{\bfseries\fontsize{8}{9.6pt}}
\renewcommand\cftsubsecfont{\fontsize{8}{9.6pt}}
\renewcommand\cftsubsubsecfont{\bfseries\fontsize{8}{9.6pt}}
\renewcommand\cftparafont{\fontsize{8}{9.6pt}}




% ########################### Glossaries ##############################
%\makeglossaries
%\newacronym{apa}{APA}{American Psychological Association}
%\newacronym{apnor}{APNOR}{Associação de Politécnicos do Norte}






